
\chapter{Pré-requis}
\minitoc
\section{Hybridation des atomes}
\subsection{Définition}
En chimie, l'hybridation des orbitales atomiques est le mélange des orbitales atomiques "conventionnelles" d'un atome appartenant à la même couche électronique de manière à former de nouvelles orbitales, dites hybrides, qui permettent de mieux décrire qualitativement les liaisons entre atomes, comme observé dans l'expérience. Les orbitales hybrides sont très utiles pour expliquer la forme des orbitales atomiques observée à travers les expérimentations.
\subsection{différents types d'orbitales atomiques}
la théorie des orbitales atomiques est née de l'expérimentation, Dans le cas idéal, les orbitales adoptent la géométrie la plus symétrique possible; d'ailleurs la symétrie est l'un des principaux facteurs induisant la dégénérescence des niveaux d'énergie, comme c'est le cas pour les orbitales hybrides qui ont toute la même énergie.
\begin{figure}[htbp]
    \begin{center}
        \includegraphics[scale=0.75]{anciennes_images/Capture_hybridation.PNG} 
    \end{center}
    \caption{Les différentes hybridations}
    \label{fig:hybridation}
\end{figure}
Pour ne pas trop rentrer dans la théorie des orbitales (n'étant pas l'objectif de ce cours) nous allons simplement indiquer comment reconnaître les hybridations des atomes à travers la formule développée ou semi-développée donnée lors des exercices, essentiellement pour l'atome de carbone (qui est, comme dit plus tôt, au centre de la chimie organique). On va résumer cela dans le tableau \ref{tab_hybridation}:
\begin{table}[!ht]
    \caption{Hybridation de l'atome de Carbone}
    \begin{center}
        \begin{tabular}{|c|c|c|}
            \hline
            \textbf{Cas} & \textbf{Hybridations} & \textbf{Formes géométriques} \\
            \hline
            Carbone avec 4 liaisons $\sigma$ & sp3 & Tétraédrique \\
            \hline
            Carbone avec 3 liaisons $\sigma$ et 1 liaison $\pi$ & sp2 & Plane (Triangle)\\
            \hline
            Carbone avec 2 liaisons $\sigma$  et 2 liaisons $\pi$ & sp & Linéaire\\
            \hline
        \end{tabular} 
    \end{center}
    \label{tab_hybridation}
\end{table}

Quelques exemples d'hybridations atomiques sont illustré dans le tableau \ref{geo_form_hybrid_ex}: 
\begin{table}[!ht]
    \caption{Forme géometrique pour chaque hybridation atomique}
    \begin{center}
        \begin{tabular}{|M{2cm}|M{5cm}|}
            \hline
            \textbf{Molécules} & \textbf{formes géométriques}  \\
            \hline
            Méthane & \includegraphics[scale=0.10]{images/ch4.png} \\
            \hline
            \'Ethène & \includegraphics[scale=0.15]{images/C2H4.png} \\
            \hline
            \'Ethyne & \includegraphics[scale=0.15]{images/C2H2.png} \\
            \hline
        \end{tabular} 
    \end{center}
    \label{geo_form_hybrid_ex}
\end{table}

\section{Classes du carbone}
La classe d'un carbone indique le nombre d'autres carbones auquel il est relié via une liaison simple covalente, on a donc 4 cas différents :
\begin{description}
    \item[Carbone primaire :] Il a une seule liaison avec un autre carbone.
    \item[Carbone secondaire :] Il a deux liaisons avec deux carbones.
    \item[Carbone tertaire :] Il a trois liaisons avec trois carbones.
    \item[Carbone quaternaire :] Il a quatre liaisons avec quatre carbones.
\end{description}

\section{Caractère nucléophile et électrophile}
\paragraph{Nucléophile}
Un nucléophile (combinaison du latin \textit{nucléus} et \textit{phile} qui veut dire aimer les noyaux et donc aimer les charges positives) est un composé chimique attiré par les espèces chargées positivement.
\paragraph{\'Electrophile}
Un composé chimique électrophile (combinaison du latin \textit{electro} et \textit{phile} qui indique une attirance à l'électricité et donc les électrons chargés négativement) est un composé chimique déficient en électrons. Il est caractérisé par sa capacité à former une liaison avec un autre composé en acceptant un doublet électronique de celui-ci il est généralement chargé positivement.

Le tableau \ref{electro_nucleo_example} illustre un certain nombre d'ions et de molécules ayant un caractère nucléophile ou électrophile.

\begin{table}[htbp]
    \caption{Exemples de certains ions et molécules nucléophiles et électrophiles}
    \begin{center}
        \begin{tabular}{|M{2cm}|M{2cm}|M{6cm}|}
            \hline
            \textbf{Caractères} & \textbf{Types} & \textbf{Exemples}\\
            \hline
            \multirow{2}{*}{Nucléophiles} & Ions & carbanion (C\textsuperscript{ -}), halogénures (X\textsuperscript{ -}), HO\textsuperscript{ -}, RO\textsuperscript{ -}, HS\textsuperscript{ -}, RS\textsuperscript{ -}, CN\textsuperscript{ -} \\ 
            \cline{2-3} 
            & Molécules & NH$_3$, amines, ROH, RSH, H$_2$O \\ 
            \hline
            \multirow{2}{*}{\'Electrophiles} & Ions &  Cations : carbocation (C\textsuperscript{ +}), H\textsuperscript{ +}, Be\textsuperscript{ 2+}, Al\textsuperscript{ 3+}\\
            \cline{2-3} 
            & Molécules &  $ BF_3 $, AlCl$_3$\\ 
            \hline
        \end{tabular}
    \end{center}
    \label{electro_nucleo_example}
\end{table}
\section{Indice d'insaturation}
Le nombre d'insaturations d'une molécule est le nombre de cycles et de liaisons multiples (doubles ou triples) qu'elle comporte.

La formule pour calculer l'indice d'insaturation (ou le degrès d'insaturation) est la suivante :
\begin{equation}
    N_i = \frac{2n_C+2 - n_H + n_N - n_X}{2}
\end{equation}
Sachant que:
\begin{description}
    \item[$n_C$ :] Le nombre d'atomes de carbone.
    \item[$n_H$  :] Le nombre d'atomes d'hydrogène.
    \item[$n_N$ :] Le nombre d'atomes d'azote.
    \item[$n_X$ :] Le nombre d'atomes d'halogène (F, Cl, Br, I).
\end{description}
L'indice d'insaturation sert justement à calculer le nombre d'insaturations sachant qu'une liaison $\pi$ ou un cycle représente une insaturation. Cet indice permet d'avoir une indication pour la représentation de la molécule à partir de la formule brute en sachant le nombre d'insaturation que la molécule comporte.

Par exemple, le Benzène: $C_6H_6$ on a:
\begin{equation}
    N_{i,C_6H_6}=\frac{2\times6+2-6}{2}=4
\end{equation}
Cela veut dire qu'il y a 4 insaturations et c'est vrai : 3 liaisons doubles + 1 cycle comme l'indique la structure du benzène représentée dans la figure \ref{fig:benzene_dev}. 
\begin{figure}[htbp]
    \centering
    \includegraphics[scale=0.1]{images/benzene.png}
    \caption{Forme développée d'un benzène}
    \label{fig:benzene_dev}
\end{figure}
\section{Acide de Lewis}
Un acide de Lewis (du nom du chimiste américain Gilbert Newton Lewis) est une entité chimique dont un des atomes la constituant possède une lacune électronique, ou case quantique vide, ce qui la rend susceptible d'accepter un doublet d'électrons, et donc de créer une liaison covalente avec une base de Lewis (donneuse de doublet électronique provenant d'un doublet non liant). Cette lacune peut être notée en représentation de Lewis par un rectangle vide comme représenté dans la figure \ref{fig:lexis_example}. Les organomagnésiens R-Mg-X (très utilisés en chimie organique) sont aussi des acides de Lewis.

Les réactions acide-base selon Lewis sont donc en fait simplement des réactions de complexation. La connaissance approfondie de ces acides n'est pas demandée dans le cadre de ce cours, on reviendra dessus au chapitre des mécanismes réactionnels.

\begin{figure}[htbp]
    \centering
    \includegraphics{images/lewis_exemple.png}
    \includegraphics[scale=0.5]{images/lewis_exemple2.png}
    \includegraphics[scale=0.75]{images/lewis_exemple3.png}
    \caption{Exemple d'acides de Lewis}
    \label{fig:lexis_example}
\end{figure}

Ou plus généralement, la figure \ref{fig:lewis_list} donne une liste des acides et bases de Lewis les plus couramment utilisés. 
\begin{figure}[htbp]
    \centering
    \includegraphics[scale=0.5]{images/lewis.png} 
    \caption{Les différents acides et bases de Lewis les plus courants}
    \label{fig:lewis_list}
\end{figure}
\section{Représentations des molécules organiques dans le plan}
Il existe différentes représentations pour une molécule organique avec plus ou moins d'informations sur cette dite molécule: le nombre d'atomes, les fonctions organiques, l'arrangement spartial de la chaîne carbonée, etc.

Les différentes représentations planes qu'on va utiliser à travers ce cours sont:
\begin{itemize}
    \item Formule développée 
    \item Formule semi développée 
    \item Formule topologique
\end{itemize}
\subsection{Formule développée}
Cette représentation donne la nature des atomes constituant la molécule ainsi que la nature des liaisons qui les unissent. Plusieurs exemples sont illustrés dans la figure \ref{fig:developpe_examples}
\begin{figure}[!ht]
    \centering
    \includegraphics[scale=0.5]{images/developpe1.png}
    \includegraphics[scale=0.3]{images/developpe2.png}
    \includegraphics[scale=0.5]{images/developpe3.png}
    \caption{Exemples de molécules représentées en formule développée}
    \label{fig:developpe_examples}
\end{figure}
\subsection{Formule semi développée}
La formule semi-développée représente toutes les liaisons de la formule développée sauf celles avec les atomes d’hydrogène. La figure \ref{fig:semi_dev_example} représente 3 acides aminés essentiels dans notre corps: L'Isoleucine, la Valine et la Leucine représentés en formule semi développée.
\begin{figure}[htbp]
    \centering
    \includegraphics[scale=0.5]{images/semideveloppe4.png}
    \caption{Isoleucine, Valine et Leucine représentés en formule semi développée}
    \label{fig:semi_dev_example}
\end{figure}
\subsection{Formule Topologique}
Cette représentation représente uniquement les liaisons carbone - carbone (avec des batonnets), les groupements significatifs (explicitement) et les liaisons avec ces groupes le reste n'est pas dessiné (donc les liaison C-H ne sont pas représentées). Plusieurs exemples sont représentés dans la figure \ref{fig:topo_example}
\begin{figure}[ht]
    \centering
    \includegraphics[scale=0.2]{"images/topo1.png"}
    \includegraphics[scale=0.35]{"images/topo2.png"}
    \includegraphics[scale=0.5]{"images/topo3.jpg"}
    \caption{Exemples de molécules représentées en formule topologique}
    \label{fig:topo_example}
\end{figure}
\begin{quote}
    \textbf{NOTE :} Les différentes représentations spatiales seront abordées lors du chapitre de stéréo-isomérie.
\end{quote}